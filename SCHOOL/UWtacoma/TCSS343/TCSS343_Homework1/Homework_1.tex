\documentclass[11pt]{article}
\title{Homework 1}
\author{Matthew Skipworth}
\date{28 June 2018}

\begin{document}
\ \\
Matthew Skipworth\\
TCSS343\\
Homework 1\\ \\
\textbf{1) }To show that $n^5-64n^3-n^2 \in \Theta (n^5)$ I chose to use the limit test:
$$\lim_{x\to\infty} \frac{n^5-64n^3-n^2}{n^5}$$
by taking the limit we see that the $n^3$ and $n^2$ terms eventually become insignificant as $n$ approaches infinity and the limit test results in $1$. 
$$\lim_{x\to\infty} \frac{n^5-64n^3-n^2}{n^5} \approx \frac{n^5}{n^5} = 1$$

\textbf{2)} We need to show that Gauss's sum $\sum_{i=1}^n i=\frac{n(n+1)}{2}$ is valid.
First we start with the base case $n=1$. $$\sum_{i=1}^1=1, \frac{1(1+1)}{2}=\frac{2}{2}=1$$ For $n=1$ Gauss's sum is true. Next we verify for $n=2$ and $n=3$. $$\sum_{i=1}^2 i=1+2=3, \frac{2(2+1)}{2}=\frac{6}{2}=3$$ $$\sum_{i=1}^3 i=1+2+3=6, \frac{3(3+1)}{2}=\frac{12}{2}=6$$ Now we infer the Inductive Hypothesis for $n=k$: $$\sum_{i=1}^k i=1+2+3+...+k=\frac{k(k+1)}{2}$$
and make the Inductive Step for n=k+1:$$\sum_{i=1}^{k+1} i=1+2+3+...+k+(k+1)=\frac{(k+1)(k+2)}{2}$$ To begin the proof, we know that the sum prior to the term $k+1$ is equal the fraction $$\frac{k(k+1)}{2}$$ so we say that the next summation is equal to the fraction plus the next term: $$\frac{k(k+1)}{2}+(k+1)$$ we then set a common denominator and combine the fraction$$\frac{k(k+1)}{2}+\frac{2(k+1)}{2}$$ $$\frac{k(k+1)+2(k+1)}{2}$$ we then factor out the $(k+1)$ term and come to our answer: $$\frac{(k+1)(k+1)}{2}$$ 
\textbf{3) }To solve this proof we first consider the base case $n=1$ $$ \sum_{i=1}^1 i^5=1, \ \bigg[ \frac{(1)((1)+1)}{2} \bigg] ^2 * \frac{2(1)^2+2(1)-1}{3} = 1$$ The statement is valid for $n=1$. Next we infer the Inductive Hypothesis for $n=k$ $$Assume\ \sum_{i=1}^k i^5 = \bigg[ \frac{k(k+1)}{2} \bigg]^2*\frac{2k^2+2k-1}{3}$$ and make the Inductive Step for n=k+1 $$\sum_{i=1}^{(k+1)} i^5 = \bigg[ \frac{k(k+1)}{2} \bigg]^2*\frac{2(k)^2+2(k)-1}{3} + (k+1)^5 = \bigg[ \frac{(k+1)(k+2)}{2} \bigg]^2*\frac{2(k+1)^2+2(k+1)-1}{3}$$ This proof was a bit of a \textit{doozy}. First we need to expand the $(k+1)^5$ term $$\bigg[ \frac{k(k+1)}{2} \bigg]^2*\frac{2k^2+2k-1}{3}+(k^5+5k^4+10k^3+10k^2+5k+1)$$ Next we expand the $ [ \frac{k(k+1)}{2} ] ^2$ term $$\bigg[\frac{k^2+k}{2}\bigg]^2*\frac{2k^2+2k-1}{3}+(k^5+5k^4+10k^3+10k^2+5k+1)$$ $$=\bigg[\frac{k^4+2k^3+k^2}{4}\bigg]*\frac{2k^2+2k-1}{3}+(k^5+5k^4+10k^3+10k^2+5k+1)$$ Now we set a common denominator of 12 and combine the fraction $$\frac{2k^6+6k^5+5k^4-k^2}{12}+\frac{12k^5+60k^4+120k^3+120k^2+60k^2+12}{12}$$ $$=\frac{2k^6+18k^5+65k^4+120k^3+119k^2+60k+12}{12}$$ Now we can factor out some terms. (I used a calculator for this part!) $$\frac{(k+1)^2(k+2)^2(2k^2+6k+3)}{12}$$ Here we break the fraction back apart $$\frac{((k+1)^2(k+2)^2)}{4}*\frac{2k^2+6k+3}{3}$$ $$=\bigg[\frac{(k+1)(k+2)}{2}\bigg]^2*\frac{2k^2+4k+2+2k+2-1}{3}$$ Finally we factor the second fraction $$=\bigg[\frac{(k+1)(k+2)}{2}\bigg]^2*\frac{2(k+1)^2+2(k+1)-1}{3}$$
and we're finished.\\ \\
\textbf{4)} We need to prove by induction that $x^n-1$ is divisible by $x-1$, for all natural numbers x and n. First we start with the base case $n=1$: $$x^1-1=x-1, \frac{(x-1)}{(x-1)}=1$$
Setting $n=1$ works. Next we try $n=2$ and $n=3$:$$x^2-1=(x-1)(x+1), \frac{(x-1)(x+1)}{(x-1)}=(x+1)$$ $$x^3-1=(x-1)(x^2+x+1), \frac{(x-1)(x^2+x+1)}{(x-1)} = x^2+x+1$$  The statement is valid for $n=2$ and $n=3$. Now we infer the Inductive Hypothesis for $n=k$ $$Assume\ x-1\ divides\ x^k-1$$ Then the Inductive Step for $n=k+1$, we must prove that $x-1$ also divides $x^{k+1}-1$. To come to a solution, we must use some algebraic tricks. We first add x to our statement and then subtract x so that the net result of our statement is unchanged. $$x^{k+1}-1+x-x$$ Next we rearrange our statement like so$$x^{k+1}-x+x-1$$ Now notice that we can group together terms such that we're left with a sum of differences$$(x^{k+1}-x)+(x-1)$$ Finally we factor an x from the first group$$x(x^{k}-1)+(x-1)$$ By the Inductive Hypothesis we see that all terms are now divisible by $x-1$.\\ \\
\textbf{5)} We need to prove by induction that $x^n-y^n$ is divisible by $x-y$ where $x \neq y$. First we must consider the base case, $n = 1$ $$ x^1-y^1=x-y,\ \frac{x-y}{x-y} = 1$$ The statement holds for $n=1$. Now we try $n=2\ and\ n=3$ $$x^2-y^2=(x+y)(x-y),\ \frac{(x+y)(x-y)}{(x-y)}=(x+y)$$ $$x^3-y^3=(x-y)(x^2+xy+y^2),\ \frac{(x-y)(x^2+xy+y^2)}{(x-y)}=x^2+xy+y^2$$ As shown above the statement holds for $n=2$ and $n=3$. Now we infer the Inductive Hypothesis for $n=k$... 
$$Assume\ x^k-y^k\ is\ divisible\ by\ x-y$$ Then the Inductive Step... $$x^{k+1}-y^{k+1}\ should\ also\ be\ divisible\ by\ x-y$$ To prove this statement we again will have to use some algebraic tricks. First notice that we can factor an x from the first term and y from the second term. $$x^{k+1}-y^{k+1}=(x)x^k-(y)y^k$$ Next we see that we can add and subtract a $y$ from the $x$ terms and add and subtract an $x$ from the $y$ terms $$(x)x^k-(y)y^k=(x+y-y)x^k-(y+x-x)y^k$$ From this equation:$$(x+y-y)x^k-(y+x-x)y^k=(x+y)()x^k-y^k)-yx^k+xy^k$$ then... $$(x+y)(x^k-y^k)+xy(x^{k-1}-y^{k-1})$$ By the Inductive Hypothesis we see that all terms are now evenly divisible by $(x-y)$.\\ \\
\textbf{6)} To show that $\sqrt[d]{n} \in \Omega (\sqrt[2d]{n})$ I chose to use the limit test.$$\lim_{n\to\infty} \frac{n^{\frac{1}{d}}}{n^{\frac{1}{2d}}}\ for\ d\ \geq 2$$ The limit can be rewritten as $$\lim_{n \to \infty} \frac{\sqrt[d]{n}}{\sqrt[2d]{n}} for\ d\ \geq 2$$ for better readability. We see here that as $n$ approaches infinity the numerator grows faster than the numerator and so...$$\lim_{n \to \infty} \frac{\sqrt[d]{n}}{\sqrt[2d]{n}} \approx \infty\ for\ d\ \geq 2$$
therefore $\sqrt[d]{n} \in \Omega \sqrt[2d]{n}$.\\ \\
To show that $(ln(n))^d \in O(\sqrt{n})$, I again used the limit test.$$\lim_{n \to \infty}\frac{(ln(n))^d}{\sqrt{n}}\ for\ d\ \geq 2$$ Taking the limit we see that$$\lim_{n \to \infty}\frac{(ln(n))^d}{\sqrt{n}}\approx\ 0\ for\ d\ \geq 2$$ Therefore $(ln(n))^d \in O(\sqrt{n})$.\\ \\
\textbf{7) }To place  the functions in the correct order(ascending) of asymptotic growth, I plotted all the functions on an online graphing tool and observed their growth rates visually.\\

$log(log(n-1)) < log(n-1) < log(n^n) < 4^{log(n)} < \frac{n}{log(n)}^3 < n^3+n^2+n+1 < 100^ {\sqrt{n}} < 9^(\frac{n}{2}) < 3^{2n}$



\end{document}